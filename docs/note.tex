\documentclass[12pt, onecolumn]{article}
\setlength\paperwidth{23cm}
% 引入相关的包
\usepackage{amsmath, listings, fontspec, geometry, graphicx, ctex, color, subfigure, amsfonts, amssymb}
\usepackage{multirow}
\usepackage[table,xcdraw]{xcolor}
\usepackage[ruled]{algorithm2e}
\usepackage[hidelinks]{hyperref}
\hypersetup{
	colorlinks=true,
	linkcolor=red,
	citecolor=red,
}

% 设定页面的尺寸和比例
\geometry{left = 1.5cm, right = 1.5cm, top = 1.5cm, bottom = 1.5cm}

% 设定两栏之间的间距
\setlength\columnsep{1cm}

% 设定字体,为代码的插入作准备
\newfontfamily\ubuntu{Ubuntu Mono}
\newfontfamily\consolas{Consolas}

% 头部信息
\title{\normf{RIC时空标定方法}}
\author{\normf{陈烁龙}}
\date{\normf{\today}}

% 代码块的风格设定
\lstset{
	language=C++,
	basicstyle=\scriptsize\ubuntu,
	keywordstyle=\textbf,
	stringstyle=\itshape,
	commentstyle=\itshape,
	numberstyle=\scriptsize\ubuntu,
	showstringspaces=false,
	numbers=left,
	numbersep=8pt,
	tabsize=2,
	frame=single,
	framerule=1pt,
	columns=fullflexible,
	breaklines,
	frame=shadowbox, 
	backgroundcolor=\color[rgb]{0.97,0.97,0.97}
}

% 字体族的定义
% \fangsong \songti \heiti \kaishu
\newcommand\normf{\fangsong}
\newcommand\boldf{\heiti}
\newcommand\keywords[1]{\boldf{关键词:} \normf #1}

\newcommand\liehat[1]{\left[ #1 \right]_\times}
\newcommand\lievee[1]{\left[ #1 \right]^\vee}
\newcommand\liehatvee[1]{\left[ #1 \right]^\vee_\times}

\newcommand\mynote[1]{{\bf{\emph{\textcolor{blue}{* \normf{#1} ...}}}}}

\newcommand\mlcomment[1]{\iffalse #1 \fi}
%\newcommand\mlcomment[1]{ #1 }

\begin{document}
	
	% 插入头部信息
	\maketitle
	% 换页
	\thispagestyle{empty}
	\clearpage
	
	% 插入目录、图、表并换页
	\pagenumbering{roman}
	\tableofcontents
	\newpage
	\listoffigures
	\newpage
	\listoftables
	% 罗马字母形式的页码
	
	\clearpage
	% 从该页开始计数
	\setcounter{page}{1}
	% 阿拉伯数字形式的页码
	\pagenumbering{arabic}
	
	
	\section{\normf{Radar Static Measurement (Without Bias)}}
	\begin{enumerate}
		\item $\{t\}$: the static target
		
		\item $\{b\}$: the body frame
		
		\item $\{b_0\}$: the first body frame
		
		\item $\tau$: the time
		
		\item $r,\theta,\phi,v_r$: the range, azimuth, elevation, and radial velocity respectively of the target with respect to the radar.
	\end{enumerate}
	The continuous-time trajectory is actually the one of the radar, we have:
	\begin{equation}
	v_r=-{^{b}\dot{\boldsymbol{p}}^\top_t(\tau)}\cdot\begin{pmatrix}
	\cos \theta\cos\phi
	\\
	\sin\theta\cos\phi
	\\
	\sin\phi
	\end{pmatrix}
	\end{equation}
	with
	\begin{equation}
	{^{b_0}\boldsymbol{p}_t}={^{b_0}_{b}\boldsymbol{R}(\tau)}\cdot{^{b}\boldsymbol{p}_t(\tau)}+{^{b_0}\boldsymbol{p}_{b}(\tau)}
	\end{equation}
	\begin{equation}
	{^{b_0}\dot{\boldsymbol{p}}_t}=\boldsymbol{0}_{3\times 1}=
	-\liehat{{^{b_0}_{b}\boldsymbol{R}(\tau)}\cdot{^{b}\boldsymbol{p}_t(\tau)}}
	\cdot{^{b_0}_{b}\dot{\boldsymbol{R}}(\tau)}
	+{^{b_0}_{b}\boldsymbol{R}(\tau)}\cdot{^{b}\dot{\boldsymbol{p}}_t(\tau)}
	+{^{b_0}\dot{\boldsymbol{p}}_{b}(\tau)}
	\end{equation}
	\begin{equation}
	{^{b_0}\dot{\boldsymbol{p}}_t}=\boldsymbol{0}_{3\times 1}=
	-{^{b_0}_{b}\boldsymbol{R}(\tau)}\cdot\liehat{{^{b}\boldsymbol{p}_t(\tau)}}
	\cdot{^{b_0}_{b}\boldsymbol{R}^\top(\tau)}\cdot{^{b_0}_{b}\dot{\boldsymbol{R}}(\tau)}
	+{^{b_0}_{b}\boldsymbol{R}(\tau)}\cdot{^{b}\dot{\boldsymbol{p}}_t(\tau)}
	+{^{b_0}\dot{\boldsymbol{p}}_{b}(\tau)}
	\end{equation}
	\begin{equation}
	{^{b_0}\dot{\boldsymbol{p}}_t}=\boldsymbol{0}_{3\times 1}=
	-\liehat{{^{b}\boldsymbol{p}_t(\tau)}}
	\cdot{^{b_0}_{b}\boldsymbol{R}^\top(\tau)}\cdot{^{b_0}_{b}\dot{\boldsymbol{R}}(\tau)}
	+{^{b}\dot{\boldsymbol{p}}_t(\tau)}
	+{^{b_0}_{b}\boldsymbol{R}^\top(\tau)}\cdot{^{b_0}\dot{\boldsymbol{p}}_{b}(\tau)}
	\end{equation}
	\begin{equation}
	{^{b}\dot{\boldsymbol{p}}_t(\tau)}=\liehat{{^{b}\boldsymbol{p}_t(\tau)}}
	\cdot{^{b_0}_{b}\boldsymbol{R}^\top(\tau)}\cdot{^{b_0}_{b}\dot{\boldsymbol{R}}(\tau)}
	-{^{b_0}_{b}\boldsymbol{R}^\top(\tau)}\cdot{^{b_0}\dot{\boldsymbol{p}}_{b}(\tau)}
	\end{equation}
	
	
	\section{\normf{Radar Static Measurement (With Bias)}}
	The continuous-time trajectory is the one of other sensor (e.g., IMU), we have:
	\begin{equation}
	{^{b_0}\boldsymbol{p}_t}={^{b_0}_{b}\boldsymbol{R}(\tau)}\cdot{^{b}\boldsymbol{p}_t(\tau)}+{^{b_0}\boldsymbol{p}_{b}(\tau)}
	\end{equation}
	with
	\begin{equation}
	{^{b}\boldsymbol{p}_t(\tau)}={^{b}_{r}\boldsymbol{R}}\cdot{^{r}\boldsymbol{p}_t(\tau)}+{^{b}\boldsymbol{p}_{r}}
	\end{equation}
	then
	\begin{equation}
	{^{b_0}\boldsymbol{p}_t}={^{b_0}_{b}\boldsymbol{R}(\tau)}\cdot{^{b}_{r}\boldsymbol{R}}\cdot{^{r}\boldsymbol{p}_t(\tau)}
	+{^{b_0}_{b}\boldsymbol{R}(\tau)}\cdot{^{b}\boldsymbol{p}_{r}}
	+{^{b_0}\boldsymbol{p}_{b}(\tau)}
	\end{equation}
	differentiate
	\begin{equation}
	{^{b_0}\dot{\boldsymbol{p}}_t}=\boldsymbol{0}_{3\times 1}=
	-\liehat{{^{b_0}_{b}\boldsymbol{R}(\tau)}\cdot{^{b}_{r}\boldsymbol{R}}\cdot{^{r}\boldsymbol{p}_t(\tau)}}\cdot{^{b_0}_{b}\dot{\boldsymbol{R}}(\tau)}
	+{^{b_0}_{b}\boldsymbol{R}(\tau)}\cdot{^{b}_{r}\boldsymbol{R}}\cdot{^{r}\dot{\boldsymbol{p}}_t(\tau)}
	-\liehat{{^{b_0}_{b}\boldsymbol{R}(\tau)}\cdot{^{b}\boldsymbol{p}_{r}}}\cdot{^{b_0}_{b}\dot{\boldsymbol{R}}(\tau)}
	+{^{b_0}\dot{\boldsymbol{p}}_{b}(\tau)}
	\end{equation}
	\begin{equation}
	\boldsymbol{0}_{3\times 1}=
	-\liehat{{^{b}_{r}\boldsymbol{R}}\cdot{^{r}\boldsymbol{p}_t(\tau)}}\cdot{^{b_0}_{b}\boldsymbol{R}^\top(\tau)}\cdot{^{b_0}_{b}\dot{\boldsymbol{R}}(\tau)}
	+{^{b}_{r}\boldsymbol{R}}\cdot{^{r}\dot{\boldsymbol{p}}_t(\tau)}
	-\liehat{{^{b}\boldsymbol{p}_{r}}}\cdot{^{b_0}_{b}\boldsymbol{R}^\top(\tau)}\cdot{^{b_0}_{b}\dot{\boldsymbol{R}}(\tau)}
	+{^{b_0}_{b}\boldsymbol{R}^\top(\tau)}\cdot{^{b_0}\dot{\boldsymbol{p}}_{b}(\tau)}
	\end{equation}
	\begin{equation}
	\boldsymbol{0}_{3\times 1}=
	-\liehat{{^{r}\boldsymbol{p}_t(\tau)}}\cdot{^{b}_{r}\boldsymbol{R}^\top}\cdot{^{b_0}_{b}\boldsymbol{R}^\top(\tau)}\cdot{^{b_0}_{b}\dot{\boldsymbol{R}}(\tau)}
	+{^{r}\dot{\boldsymbol{p}}_t(\tau)}
	-{^{b}_{r}\boldsymbol{R}^\top}\cdot\liehat{{^{b}\boldsymbol{p}_{r}}}\cdot{^{b_0}_{b}\boldsymbol{R}^\top(\tau)}\cdot{^{b_0}_{b}\dot{\boldsymbol{R}}(\tau)}
	+{^{b}_{r}\boldsymbol{R}^\top}\cdot{^{b_0}_{b}\boldsymbol{R}^\top(\tau)}\cdot{^{b_0}\dot{\boldsymbol{p}}_{b}(\tau)}
	\end{equation}
	thus
	\begin{equation}
	{^{r}\dot{\boldsymbol{p}}_t(\tau)}=
	\liehat{{^{r}\boldsymbol{p}_t(\tau)}}\cdot{^{b}_{r}\boldsymbol{R}^\top}\cdot{^{b_0}_{b}\boldsymbol{R}^\top(\tau)}\cdot{^{b_0}_{b}\dot{\boldsymbol{R}}(\tau)}
	+{^{b}_{r}\boldsymbol{R}^\top}\cdot\liehat{{^{b}\boldsymbol{p}_{r}}}\cdot{^{b_0}_{b}\boldsymbol{R}^\top(\tau)}\cdot{^{b_0}_{b}\dot{\boldsymbol{R}}(\tau)}
	-{^{b}_{r}\boldsymbol{R}^\top}\cdot{^{b_0}_{b}\boldsymbol{R}^\top(\tau)}\cdot{^{b_0}\dot{\boldsymbol{p}}_{b}(\tau)}
	\end{equation}
	
	
	
	
	
	
	
	
	
	
	
	
	
	
	
	
	
	
\end{document}

