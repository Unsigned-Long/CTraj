\documentclass[12pt, onecolumn]{article}
\setlength\paperwidth{23cm}
% 引入相关的包
\usepackage{amsmath, listings, fontspec, geometry, graphicx, ctex, color, subfigure, amsfonts, amssymb}
\usepackage{multirow}
\usepackage[table,xcdraw]{xcolor}
\usepackage[ruled]{algorithm2e}
\usepackage[hidelinks]{hyperref}
\hypersetup{
	colorlinks=true,
	linkcolor=red,
	citecolor=red,
}

% 设定页面的尺寸和比例
\geometry{left = 1.5cm, right = 1.5cm, top = 1.5cm, bottom = 1.5cm}

% 设定两栏之间的间距
\setlength\columnsep{1cm}

% 设定字体,为代码的插入作准备
\newfontfamily\ubuntu{Ubuntu Mono}
\newfontfamily\consolas{Consolas}

% 头部信息
\title{\normf{RIC时空标定方法}}
\author{\normf{陈烁龙}}
\date{\normf{\today}}

% 代码块的风格设定
\lstset{
	language=C++,
	basicstyle=\scriptsize\ubuntu,
	keywordstyle=\textbf,
	stringstyle=\itshape,
	commentstyle=\itshape,
	numberstyle=\scriptsize\ubuntu,
	showstringspaces=false,
	numbers=left,
	numbersep=8pt,
	tabsize=2,
	frame=single,
	framerule=1pt,
	columns=fullflexible,
	breaklines,
	frame=shadowbox, 
	backgroundcolor=\color[rgb]{0.97,0.97,0.97}
}

% 字体族的定义
% \fangsong \songti \heiti \kaishu
\newcommand\normf{\fangsong}
\newcommand\boldf{\heiti}
\newcommand\keywords[1]{\boldf{关键词:} \normf #1}

\newcommand\liehat[1]{\left[ #1 \right]_\times}
\newcommand\lievee[1]{\left[ #1 \right]^\vee}
\newcommand\liehatvee[1]{\left[ #1 \right]^\vee_\times}

\newcommand\mynote[1]{{\bf{\emph{\textcolor{blue}{* \normf{#1} ...}}}}}

\newcommand\mlcomment[1]{\iffalse #1 \fi}
%\newcommand\mlcomment[1]{ #1 }

\newcommand\bsm[1]{\boldsymbol{\mathrm{#1}}}
\newcommand\rotation[2]{{\bsm{R}_{#1}^{#2}}}
\newcommand\quaternion[2]{{\bsm{q}_{#1}^{#2}}}
\newcommand\angvel[2]{{\bsm{\omega}_{#1}^{#2}}}
\newcommand\translation[2]{{\bsm{p}_{#1}^{#2}}}
\newcommand\translationhat[2]{{\hat{\bsm{p}}_{#1}^{#2}}}
\newcommand\linvel[2]{{\bsm{v}_{#1}^{#2}}}
\newcommand\linacce[2]{{\bsm{a}_{#1}^{#2}}}
\newcommand\gravity[1]{{\bsm{g}^{#1}}}
\newcommand\smallminus{{\text{-}}}
\newcommand\smallplus{{\text{+}}}
\newcommand\coordframe[1]{\underrightarrow{\mathcal{F}}_{#1}}

\begin{document}
	
	% 插入头部信息
	\maketitle
	% 换页
	\thispagestyle{empty}
	\clearpage
	
	% 插入目录、图、表并换页
	\pagenumbering{roman}
	\tableofcontents
	\newpage
	\listoffigures
	\newpage
	\listoftables
	% 罗马字母形式的页码
	
	\clearpage
	% 从该页开始计数
	\setcounter{page}{1}
	% 阿拉伯数字形式的页码
	\pagenumbering{arabic}
	
	
	\section{\normf{Radar Static Measurement (V1)}}
	The continuous-time trajectory is the one of other sensor (e.g., IMU), we have:
	\begin{equation}
	{^{b_0}\boldsymbol{p}_t}={^{b_0}_{b}\boldsymbol{R}(\tau)}\cdot{^{b}\boldsymbol{p}_t(\tau)}+{^{b_0}\boldsymbol{p}_{b}(\tau)}
	\end{equation}
	with
	\begin{equation}
	{^{b}\boldsymbol{p}_t(\tau)}={^{b}_{r}\boldsymbol{R}}\cdot{^{r}\boldsymbol{p}_t(\tau)}+{^{b}\boldsymbol{p}_{r}}
	\end{equation}
	then
	\begin{equation}
	{^{b_0}\boldsymbol{p}_t}={^{b_0}_{b}\boldsymbol{R}(\tau)}\cdot{^{b}_{r}\boldsymbol{R}}\cdot{^{r}\boldsymbol{p}_t(\tau)}
	+{^{b_0}_{b}\boldsymbol{R}(\tau)}\cdot{^{b}\boldsymbol{p}_{r}}
	+{^{b_0}\boldsymbol{p}_{b}(\tau)}
	\end{equation}
	differentiate
	\begin{equation}
	{^{b_0}\dot{\boldsymbol{p}}_t}=\boldsymbol{0}_{3\times 1}=
	-\liehat{{^{b_0}_{b}\boldsymbol{R}(\tau)}\cdot{^{b}_{r}\boldsymbol{R}}\cdot{^{r}\boldsymbol{p}_t(\tau)}}\cdot{^{b_0}_{b}\dot{\boldsymbol{R}}(\tau)}
	+{^{b_0}_{b}\boldsymbol{R}(\tau)}\cdot{^{b}_{r}\boldsymbol{R}}\cdot{^{r}\dot{\boldsymbol{p}}_t(\tau)}
	-\liehat{{^{b_0}_{b}\boldsymbol{R}(\tau)}\cdot{^{b}\boldsymbol{p}_{r}}}\cdot{^{b_0}_{b}\dot{\boldsymbol{R}}(\tau)}
	+{^{b_0}\dot{\boldsymbol{p}}_{b}(\tau)}
	\end{equation}
	\begin{equation}
	\boldsymbol{0}_{3\times 1}=
	-\liehat{{^{b}_{r}\boldsymbol{R}}\cdot{^{r}\boldsymbol{p}_t(\tau)}}\cdot{^{b_0}_{b}\boldsymbol{R}^\top(\tau)}\cdot{^{b_0}_{b}\dot{\boldsymbol{R}}(\tau)}
	+{^{b}_{r}\boldsymbol{R}}\cdot{^{r}\dot{\boldsymbol{p}}_t(\tau)}
	-\liehat{{^{b}\boldsymbol{p}_{r}}}\cdot{^{b_0}_{b}\boldsymbol{R}^\top(\tau)}\cdot{^{b_0}_{b}\dot{\boldsymbol{R}}(\tau)}
	+{^{b_0}_{b}\boldsymbol{R}^\top(\tau)}\cdot{^{b_0}\dot{\boldsymbol{p}}_{b}(\tau)}
	\end{equation}
	\begin{equation}
	\boldsymbol{0}_{3\times 1}=
	-\liehat{{^{r}\boldsymbol{p}_t(\tau)}}\cdot{^{b}_{r}\boldsymbol{R}^\top}\cdot{^{b_0}_{b}\boldsymbol{R}^\top(\tau)}\cdot{^{b_0}_{b}\dot{\boldsymbol{R}}(\tau)}
	+{^{r}\dot{\boldsymbol{p}}_t(\tau)}
	-{^{b}_{r}\boldsymbol{R}^\top}\cdot\liehat{{^{b}\boldsymbol{p}_{r}}}\cdot{^{b_0}_{b}\boldsymbol{R}^\top(\tau)}\cdot{^{b_0}_{b}\dot{\boldsymbol{R}}(\tau)}
	+{^{b}_{r}\boldsymbol{R}^\top}\cdot{^{b_0}_{b}\boldsymbol{R}^\top(\tau)}\cdot{^{b_0}\dot{\boldsymbol{p}}_{b}(\tau)}
	\end{equation}
	thus, the velocity of target $\{t\}$ with respect to the radar $\{r\}$ parameterized in $\{r\}$ could be expressed as:
	\begin{equation}
	{^{r}\dot{\boldsymbol{p}}_t(\tau)}=
	\liehat{{^{r}\boldsymbol{p}_t(\tau)}}\cdot{^{b}_{r}\boldsymbol{R}^\top}\cdot{^{b_0}_{b}\boldsymbol{R}^\top(\tau)}\cdot{^{b_0}_{b}\dot{\boldsymbol{R}}(\tau)}
	+{^{b}_{r}\boldsymbol{R}^\top}\cdot\liehat{{^{b}\boldsymbol{p}_{r}}}\cdot{^{b_0}_{b}\boldsymbol{R}^\top(\tau)}\cdot{^{b_0}_{b}\dot{\boldsymbol{R}}(\tau)}
	-{^{b}_{r}\boldsymbol{R}^\top}\cdot{^{b_0}_{b}\boldsymbol{R}^\top(\tau)}\cdot{^{b_0}\dot{\boldsymbol{p}}_{b}(\tau)}
	\end{equation}
	
	\section{\normf{Radar Static Measurement (V2)}}
	\begin{equation}
	{^{b_0}{\boldsymbol{p}}_r(\tau)}={^{b_0}_{b}\boldsymbol{R}(\tau)}\cdot{^{b}{\boldsymbol{p}}_r}+{^{b_0}{\boldsymbol{p}}_b(\tau)}
	\end{equation}
	\begin{equation}
	{^{b_0}\dot{\boldsymbol{p}}_r(\tau)}=-\liehat{{^{b_0}_{b}\boldsymbol{R}(\tau)}\cdot{^{b}{\boldsymbol{p}}_r}}\cdot{^{b_0}_{b}\dot{\boldsymbol{R}}(\tau)}
	+{^{b_0}\dot{\boldsymbol{p}}_b(\tau)}
	\end{equation}
	thus, the velocity of radar $\{r\}$ with respect to the frame $\{b_0\}$ parameterized in $\{r\}$ could be expressed as:
	\begin{equation}
	{^{b}_{r}\boldsymbol{R}^\top}\cdot{^{b_0}_{b}\boldsymbol{R}^\top}\cdot{^{b_0}\dot{\boldsymbol{p}}_r(\tau)}=
	{^{b}_{r}\boldsymbol{R}^\top}\cdot{^{b_0}_{b}\boldsymbol{R}^\top}\cdot
	\left(
	-\liehat{{^{b_0}_{b}\boldsymbol{R}(\tau)}\cdot{^{b}{\boldsymbol{p}}_r}}\cdot{^{b_0}_{b}\dot{\boldsymbol{R}}(\tau)}
	+{^{b_0}\dot{\boldsymbol{p}}_b(\tau)}
	\right) 
	\end{equation}
	actually, by introducing $\liehat{{^{r}\boldsymbol{p}_t(\tau)}}$, we have:
	\begin{equation}
	\liehat{{^{r}\boldsymbol{p}_t(\tau)}}\cdot{^{b}_{r}\boldsymbol{R}^\top}\cdot{^{b_0}_{b}\boldsymbol{R}^\top}\cdot{^{b_0}\dot{\boldsymbol{p}}_r(\tau)}=-\liehat{{^{r}\boldsymbol{p}_t(\tau)}}\cdot{^{r}\dot{\boldsymbol{p}}_t(\tau)}
	\end{equation}
	
	\section{\normf Preintegration}
	\normf
	IMU的坐标系记为$\coordframe{b}$,世界坐标系记为$\coordframe{w}$,则有:
	\begin{equation}
	\begin{aligned}
	\bsm{a}_\tau&=\rotation{b_\tau}{w} ^\top\cdot\left(
	\linacce{b_\tau}{w}-\gravity{w}
	\right) 
	\\
	\bsm{\omega}_\tau&=\rotation{b_\tau}{w} ^\top\cdot\angvel{b_\tau}{w}
	\end{aligned}
	\end{equation}
	通过变换,得到:
	\begin{equation}
	\begin{aligned}
	\linacce{b_\tau}{w}&=\rotation{b_\tau}{w} \cdot \bsm{a}_\tau+\gravity{w}
	\\
	\angvel{b_\tau}{w}&=\rotation{b_\tau}{w}\cdot\bsm{\omega}_\tau
	\end{aligned}
	\end{equation}
	通过积分得到:
	\begin{equation}
	\begin{aligned}
		\linvel{b_{k+1}}{w}&=\linvel{b_{k}}{w}+\gravity{w}\cdot\Delta\tau
		+\int_{\tau_{k}}^{\tau_{k+1}}
		\rotation{b_t}{w} \cdot \bsm{a}_t \cdot dt
		\\
		\translation{b_{k+1}}{w}&=\translation{b_{k}}{w}
		+\linvel{b_{k}}{w}\cdot\Delta\tau
		+\frac{1}{2}\cdot\gravity{w}\cdot\Delta\tau^2+\iint_{\tau_{k}}^{\tau_{k+1}}
		\rotation{b_t}{w} \cdot \bsm{a}_t \cdot dt^2
		\\
		\quaternion{b_{k+1}}{w}&=\quaternion{b_k}{w}\circ
		 \int_{\tau_{k}}^{\tau_{k+1}}\frac{1}{2}\cdot\boldsymbol{\Omega}(\bsm{\omega}_t)\cdot\quaternion{b_t}{b_k}\cdot dt
	\end{aligned}
	\end{equation}
	其中:
	\begin{equation}
	\boldsymbol{\Omega}(\bsm{\omega}_t)=\begin{bmatrix}
	-\liehat{\bsm{\omega}_t} & \bsm{\omega}_t\\
	-\bsm{\omega}_t^\top & 0
	\end{bmatrix}
	\end{equation}
	将参考坐标系变换到$\coordframe{b_k}$,得到:
	\begin{equation}
	\begin{aligned}
		\rotation{b_{k}}{w}^\top\cdot\linvel{b_{k+1}}{w}&=\rotation{b_{k}}{w}^\top\cdot\left( \linvel{b_{k}}{w}+\gravity{w}\cdot\Delta\tau\right) 
		+\int_{\tau_{k}}^{\tau_{k+1}}
		\rotation{b_t}{b_{k}} \cdot \bsm{a}_t \cdot dt
		\\
		\rotation{b_{k}}{w}^\top\cdot\translation{b_{k+1}}{w}&=\rotation{b_{k}}{w}^\top\cdot\left( \translation{b_{k}}{w}
		+\linvel{b_{k}}{w}\cdot\Delta\tau
		+\frac{1}{2}\cdot\gravity{w}\cdot\Delta\tau^2\right) +\iint_{\tau_{k}}^{\tau_{k+1}}
		\rotation{b_t}{b_{k}} \cdot \bsm{a}_t \cdot dt^2
		\\
		\left( \quaternion{b_k}{w}\right) ^{-1}\circ\quaternion{b_{k+1}}{w}&=
		 \int_{\tau_{k}}^{\tau_{k+1}}\frac{1}{2}\cdot\boldsymbol{\Omega}(\bsm{\omega}_t)\cdot\quaternion{b_t}{b_k}\cdot dt
	\end{aligned}
	\end{equation}
	得到预积分项,并带入带有噪声的观测值:
	\begin{equation}
	\begin{aligned}
	\bsm{\alpha}_{b_{k+1}}^{b_k}&=\iint_{\tau_{k}}^{\tau_{k+1}}
				\rotation{b_t}{b_{k}} \cdot \left( \hat{\bsm{a}}_t-\bsm{b}_{a_t}-\bsm{n}_a\right) \cdot dt^2
				\\
	\bsm{\beta}_{b_{k+1}}^{b_k}&=\int_{\tau_{k}}^{\tau_{k+1}}
			\rotation{b_t}{b_{k}} \cdot \left( \hat{\bsm{a}}_t-\bsm{b}_{a_t}-\bsm{n}_a\right)  \cdot dt
			\\
	\bsm{\gamma}_{b_{k+1}}^{b_k}&=\int_{\tau_{k}}^{\tau_{k+1}}\frac{1}{2}\cdot\boldsymbol{\Omega}(\hat{\bsm{\omega}}_t-\bsm{b}_{g_t}-\bsm{n}_g)\cdot\quaternion{b_t}{b_k}\cdot dt
	\end{aligned}
	\end{equation}
	数值积分使用midpoint进行:
	\begin{equation}
	\begin{aligned}
	\bsm{\alpha}_{b_{j+1}}^{b_k}&=\bsm{\alpha}_{b_{j}}^{b_k}+\bsm{\beta}_{b_{j}}^{b_k}\cdot\delta \tau+\frac{1}{2}\cdot\rotation{b_j}{b_k}\cdot\left( \hat{\bsm{a}}_j-\bsm{b}_{a_j}\right) \cdot\delta \tau^2
	\\
	\bsm{\beta}_{b_{j+1}}^{b_k}&=\bsm{\beta}_{b_{j}}^{b_k}+\rotation{b_j}{b_k}\cdot\left( \hat{\bsm{a}}_j-\bsm{b}_{a_j}\right) \cdot\delta \tau
	\\
	\bsm{\gamma}_{b_{j+1}}^{b_k}&=\bsm{\gamma}_{b_{j}}^{b_k}\circ\begin{bmatrix}
	1\\\begin{aligned}
	\frac{1}{2}\cdot\left(\hat{\bsm{\omega}}_j-\bsm{b}_{g_j} \right) \cdot\delta \tau
	\end{aligned}
	\end{bmatrix}
	\end{aligned}
	\end{equation}
	推导方差传播模型:
	\begin{equation}
	\begin{bmatrix}
	\delta\dot{\bsm{\alpha}}_{b_{t}}^{b_k}
	\\
	\delta\dot{\bsm{\beta}}_{b_{t}}^{b_k}
	\\
	\delta\dot{\bsm{\theta}}_{b_{t}}^{b_k}
	\\
	\delta\dot{\bsm{b}}_{a_{t}}
	\\
	\delta\dot{\bsm{b}}_{g_{t}}
	\end{bmatrix}=
	\begin{bmatrix}
	0&\bsm{I}_3&0&0&0\\
	0&0&-\rotation{b_t}{b_{k}}\cdot\liehat{\hat{\bsm{a}}_t-\bsm{b}_{a_t}}&-\rotation{b_t}{b_{k}}&0\\
	0&0&-\liehat{\hat{\bsm{\omega}}_t-\bsm{b}_{g_t}}&0&-\bsm{I}_3\\
	0&0&0&0&0\\0&0&0&0&0
	\end{bmatrix}
	\begin{bmatrix}
	\delta{\bsm{\alpha}}_{b_{t}}^{b_k}
	\\
	\delta{\bsm{\beta}}_{b_{t}}^{b_k}
	\\
	\delta{\bsm{\theta}}_{b_{t}}^{b_k}
	\\
	\delta{\bsm{b}}_{a_{t}}
	\\
	\delta{\bsm{b}}_{g_{t}}
	\end{bmatrix}+
	\begin{bmatrix}
	0&0&0&0\\
	-\rotation{b_t}{b_{k}}&0&0&0\\
	0&-\bsm{I}_3&0&0\\
	0&0&\bsm{I}_3&0\\
	0&0&0&\bsm{I}_3
	\end{bmatrix}\begin{bmatrix}
	\bsm{n}_a\\\bsm{n}_g\\\bsm{n}_{b_a}\\\bsm{n}_{b_g}
	\end{bmatrix}=\bsm{F}_t\delta\bsm{z}^{b_k}_{t}+\bsm{G}_{t}\bsm{n}_{t}
	\end{equation}
	注意,对旋转矩阵的求导为右扰动,所以对应的姿态更新为:
	\begin{equation}
	\bsm{\gamma}_{b_{t}}^{b_k}=\bsm{\gamma}_{b_{t}}^{b_k}\circ\delta\bsm{\gamma}
	\end{equation}
	进行离散化,得到:
	\begin{equation}
	\bsm{z}^{b_k}_{t_2}=\left( \bsm{I}_5+\bsm{F}_t\cdot\delta\tau\right)\cdot\bsm{z}^{b_k}_{t_1} +
	\int_{0}^{\delta \tau}\bsm{F}_t\cdot\bsm{G}_t\cdot\bsm{Q}\cdot\bsm{G}_t^\top\cdot\bsm{F}_t^\top
	\end{equation}
	之后进行滤波即可。
	
	
	
	
	
	
	
	
	
	
	
	
	
	
\end{document}

